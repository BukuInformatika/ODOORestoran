
\documentclass{wileySix}

\usepackage{graphicx}
\usepackage{listings}

\usepackage{color}

\definecolor{codegreen}{rgb}{0,0.6,0}
\definecolor{codegray}{rgb}{0.5,0.5,0.5}
\definecolor{codepurple}{rgb}{0.58,0,0.82}
\definecolor{backcolour}{rgb}{0.95,0.95,0.92}

\lstdefinestyle{mystyle}{
    backgroundcolor=\color{backcolour},
    commentstyle=\color{codegreen},
    keywordstyle=\color{magenta},
    numberstyle=\tiny\color{codegray},
    stringstyle=\color{codepurple},
    basicstyle=\footnotesize,
    breakatwhitespace=false,
    breaklines=true,
    captionpos=b,
    keepspaces=true,
    numbers=left,
    numbersep=5pt,
    showspaces=false,
    showstringspaces=false,
    showtabs=false,
    tabsize=2,
    language=sh
}

\lstset{style=mystyle}

\usepackage{w-bookps}

\newcommand{\VT}[1]{\ensuremath{{V_{T#1}}}}

\newbox\sectsavebox
\setbox\sectsavebox=\hbox{\boldmath\VT{xyz}}

\begin{document}


\booktitle{ODOO Restoran}
\subtitle{ODOO Restaurant Management}

\authors{Rolly M. Awangga\\
\affil{Informatics Research Center}

\offprintinfo{ODOO Restoran, First Edition}{Rolly M. Awangga}

\halftitlepage

\titlepage


\begin{copyrightpage}{2019}
%Survey Methodology / Robert M. Groves . . . [et al.].
%\       p. cm.---(Wiley series in survey methodology)
%\    ``Wiley-Interscience."
%\    Includes bibliographical references and index.
%\    ISBN 0-471-48348-6 (pbk.)
%\    1. Surveys---Methodology.  2. Social 
%\  sciences---Research---Statistical methods.  I. Groves, Robert M.  II. %
%Series.\\
%
%HA31.2.S873 2007
%001.4'33---dc22                                             2004044064
\end{copyrightpage}

\dedication{`Jika Kamu tidak dapat menahan lelahnya belajar,
Maka kamu harus sanggup menahan perihnya Kebodohan.'
~Imam Syafi'i~}

\begin{contributors}
\name{Rolly Maulana Awangga,} Informatics Research Center., Politeknik Pos Indonesia, Bandung,
Indonesia



\end{contributors}

\contentsinbrief
\tableofcontents
\listoffigures
\listoftables
\lstlistoflistings


\begin{foreword}
Sepatah kata dari Kaprodi, Kabag Kemahasiswaan dan Mahasiswa
\end{foreword}

\begin{preface}
Buku ini diciptakan bagi yang awam dengan git sekalipun.

\prefaceauthor{R. M. Awangga}
\where{Bandung, Jawa Barat\\
Februari, 2019}
\end{preface}


\begin{acknowledgments}
Terima kasih atas semua masukan dari para mahasiswa agar bisa membuat buku ini 
lebih baik dan lebih mudah dimengerti.

Terima kasih ini juga ditujukan khusus untuk team IRC yang 
telah fokus untuk belajar dan memahami bagaimana buku ini mendampingi proses 
Intership.
\authorinitials{R. M. A.}
\end{acknowledgments}

\begin{acronyms}
\acro{ACGIH}{American Conference of Governmental Industrial Hygienists}
\acro{AEC}{Atomic Energy Commission}
\acro{OSHA}{Occupational Health and Safety Commission}
\acro{SAMA}{Scientific Apparatus Makers Association}
\end{acronyms}

\begin{glossary}
\term{git}Merupakan manajemen sumber kode yang dibuat oleh linus torvald.

\term{bash}Merupakan bahasa sistem operasi berbasiskan *NIX.

\term{linux}Sistem operasi berbasis sumber kode terbuka yang dibuat oleh Linus Torvald
\end{glossary}

\begin{symbols}
\term{A}Amplitude

\term{\hbox{\&}}Propositional logic symbol 

\term{a}Filter Coefficient

\bigskip

\term{\mathcal{B}}Number of Beats
\end{symbols}

\begin{introduction}

%% optional, but if you want to list author:

\introauthor{Rolly Maulana Awangga, S.T., M.T.}
{Informatics Research Center\\
Bandung, Jawa Barat, Indonesia}

Pada era disruptif  \index{disruptif}\index{disruptif!modern} 
saat ini. git merupakan sebuah kebutuhan dalam sebuah organisasi pengembangan perangkat lunak.
Buku ini diharapkan bisa menjadi penghantar para programmer, analis, IT Operation dan Project Manajer.
Dalam melakukan implementasi git pada diri dan organisasinya.

Rumusnya cuman sebagai contoh aja biar keren\cite{awangga2018sampeu}.

\begin{equation}
ABC {\cal DEF} \alpha\beta\Gamma\Delta\sum^{abc}_{def}
\end{equation}

\end{introduction}

%%%%%%%%%%%%%%%%%%Isi Buku_

\chapter{Pengenalan}
\section{Pengenalan}
ODOO adalah \textit{all-in-one software} berbasis \textit{open-source} yang mempunyai fitur \textit{ CRM, website/e-commerce, billing, accounting, manufacturing, warehouse - and project management,} dan \textit{inventory}.
\section{Manajemen Restoran Menggunakan ODOO}
ODOO sangatlah cocok untuk \textit{western restaurant}. Karena fitur yang akan digunakan yaitu \textit{Table Management, Bartenders System, Bankers System, PTC,} dan lainnya. Sistem Restoran menggunakan ODOO ini sudah terbukti digunakan di \textit{Coffee Shop} di Boston, Amerika Serikat.

\chapter{Judul Bagian Kedua}
\section{Instalasi ODOO di Windows}
Masuk ke situs https://www.odoo.com/page/download lalu disitu ada 2 pilihan
\begin{itemize}
  \item Mengisi form mengenai nama Perusahaan, nama Anda, dan-lain lain sebelum download.
  \item Mencoba tanpa harus download.
\end{itemize}
Disitu juga terdapat 2 versi download, versi Community dan Enterprise. Berikut akan dijelaskan satu per satu:



\bibliographystyle{IEEEtran}
%\def\bibfont{\normalsize}
\bibliography{references}


%%%%%%%%%%%%%%%
%%  The default LaTeX Index
%%  Don't need to add any commands before \begin{document}
\printindex

%%%% Making an index
%%
%% 1. Make index entries, don't leave any spaces so that they
%% will be sorted correctly.
%%
%% \index{term}
%% \index{term!subterm}
%% \index{term!subterm!subsubterm}
%%
%% 2. Run LaTeX several times to produce <filename>.idx
%%
%% 3. On command line, type  makeindx <filename> which
%% will produce <filename>.ind
%%
%% 4. Type \printindex to make the index appear in your book.
%%
%% 5. If you would like to edit <filename>.ind
%% you may do so. See docs.pdf for more information.
%%
%%%%%%%%%%%%%%%%%%%%%%%%%%%%%%

%%%%%%%%%%%%%% Making Multiple Indices %%%%%%%%%%%%%%%%
%% 1.
%% \usepackage{multind}
%% \makeindex{book}
%% \makeindex{authors}
%% \begin{document}
%%
%% 2.
%% % add index terms to your book, ie,
%% \index{book}{A term to go to the topic index}
%% \index{authors}{Put this author in the author index}
%%
%% \index{book}{Cows}
%% \index{book}{Cows!Jersey}
%% \index{book}{Cows!Jersey!Brown}
%%
%% \index{author}{Douglas Adams}
%% \index{author}{Boethius}
%% \index{author}{Mark Twain}
%%
%% 3. On command line type
%% makeindex topic
%% makeindex authors
%%
%% 4.
%% this is a Wiley command to make the indices print:
%% \multiprintindex{book}{Topic index}
%% \multiprintindex{authors}{Author index}

\end{document}

